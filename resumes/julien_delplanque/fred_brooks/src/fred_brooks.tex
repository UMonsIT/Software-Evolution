\documentclass[a4paper,11pt]{report}

\usepackage[utf8]{inputenc}
\usepackage[T1]{fontenc}
\usepackage[francais]{babel}
\usepackage{amsfonts}
\usepackage{amsmath}
\usepackage{amsthm}
\usepackage{listings}
\usepackage{fullpage}

\renewcommand{\thesection}{\arabic{section}}

\newtheorem{definition}{Definition}

\title{A view on software evolution by Fred Books: Summary}
\author{Julien Delplanque}
\begin{document}
\maketitle
\newpage

\section{Chapter 11: ``Plan to Throw One Away''}
This chapter emphasises the inevitability of changes and the need to accommodate
these changes in: the development process, the software product and the
organisation.\\

\textbf{The only constancy is change itself.}
Programs are developed to satisfy a user's needs but this need itself, and the
user's perception change as programs are built, tested and used. Also, due to
the very nature of software (software are soft), they are more exposed to
changes in requirements than other products. Nevertheless, not all user requests
for changes in requirements need to be incorporated; a mechanism to decide which
change request to accept is needed and the threshold for accepting changes must
become higher as development proceeds.\\

\textbf{Changes are inevitable.} They occurs in the user requirements and
objectives, in the development strategy, in the technology used and also in the
company structure (people come and go).\\

\textbf{The system should be plan for change} and there are techniques to do it:
using well-known programming principles such as modularisation and
encapsulation, low coupling and high cohesion and good documentation; using good
change control mechanisms such as a versioning system with each version having
its own fixed shedule.\\

\textbf{The organisation should be plan for change.} In a project schedule,
change should be allowed and accommodated in a controlled way: milestones,
deadline, and so one may be changed along the way as long as the impact of these
changes is analysed (regular revisions of project plan whenever some task
experiences a delay). People should be allowed to advance in their career by
making them interchangeable as possible (to avoid project failure when one key
person leaves). Managers and technical people should be interchangeable.
Promotion to a higher level should take into account individual talents. The
preceding guidelines minimise the risk that organisational changes leads to
project failure.\\

\textbf{Software that has been delivered continues to change.} This, even if no
functionality is added. ``Maintenance'' is needed to fix bugs and repair
defects. ``Software maintenance'' is different from ``hardware maintenance''
because as opposed to hardware, software itself does not deteriorate but becomes
less useful in a changing environment. Therefore, besides repairing defects, it
is therefore essential to add functionality to keep the software ``useful''.\\

\textbf{Software maintenance costs more than $40\%$ of software development
costs} and the more users a software has, the more bugs are discovered. During
the software usage life cycle, bugs gradually decrease, then start to increase
again (Because users start to use full capabilities of the system?). Fixing bugs
is likely to introduce new bugs (often, suboptimal patches are made).

\clearpage
\section{Essence of software engineering}

\begin{definition}[Essence of software engineering]
The ``essence'' of software engineering corresponds to the difficulties that are
inherent in the nature of software. This ``essence'' cannot be reduced and
appears under $4$ different forms:
\begin{enumerate}
\item Complexity,
\item Conformity,
\item Changeability,
\item Invisibility.
\end{enumerate}
\end{definition}

\subsection{Complexity}
Software is more \textbf{complex} than any other human-engineered artefact.
This is due to the fact that abstraction (e.g. modeling) can be used to hide
complexity byt not to avoid it. Software complexity increases non-linearly with
its size and implies managerial complexity.

\subsection{Conformity}
Software must \textbf{conform} to its environment because it is employed in the
``real'' world, so it must deal with this complexity of this world:
interoperability with other systems (hardware or software), interaction with
human actors, conformance to industry and company standards, conformance to
legislation, (backward) compatibility with languages, conformity with
technology, etc\dots

\subsection{Changeability}
Software is constantly subject to pressure for \textbf{changes} (much more than
other engineered artefacts like cars for example) because major architectural
changes are still possible after deployment. Also, changes have a very high cost
and are inevitable (all successful software gets changed to invent new uses for
it or to survive the life of the machine for which it was originally written).

\subsection{Invisibility}
It is difficult to \textbf{visualize} software.

\end{document}
